\chapter{Problem}
In this work, we have to create a model of 2 planar connected using tendon.

\begin{figure}[H]
\centering
\includegraphics[width=1.0\linewidth]{images/problem}
\caption{Schema from the task}
\label{fig:problem}
\end{figure}

My variant for the problem is shown below:
\begin{table}[h!]
\centering
\begin{tabular}{|c|c|c|c|c|c|}
\hline
R1 & R2 & a & b & c \\ \hline
0.021 &	0.017 &	0.045 &	0.099 &	0.043 \\ \hline
\end{tabular}
\caption{System parameters}
\label{tab:system_parameters}
\end{table}

Then we have to add the actuators which attached to the tendons with the following parameters:
\begin{table}[h!]
\centering
\begin{tabular}{|c|c|c|c|}
\hline
q & Amplitude & Frequency & Bias \\ \hline
1 & 3.616 & 36.16 & 3.21 \\ \hline
2 & 4.97 & 2.24 & -3.58 \\ \hline
\end{tabular}
\caption{Parameters for actuators}
\end{table}

\chapter{Solution}

The model can be viewed in the included XML file.

We use this helper function to set the torque for the actuator.
\begin{lstlisting}[language=Python]
def set_torque(mj_data, actuator:int, time, a, f, p):
    mj_data.ctrl[actuator] = a * np.sin(time * f + p)
\end{lstlisting}

\begin{lstlisting}[language=Python]
...
AMP_1 = 3.616
FREQ_1 = 36.16
BIAS_1 = 3.21

AMP_2 = 4.97
FREQ_2 = 2.24
BIAS_2 = -3.58
...

for i in range(STEP_NUM):  
    if viewer.is_alive:     
        set_torque(data, 0, data.time, AMP_1, FREQ_1, BIAS_1)
        set_torque(data, 1, data.time, AMP_2, FREQ_2, BIAS_2)
        
        ...

    else:
        break
viewer.close()

\end{lstlisting}

\chapter{Result}
\begin{figure}[H]
\centering
\includegraphics[width=1.0\linewidth]{images/img1}
\caption{}
\label{fig:img1}
\end{figure}


\begin{figure}[H]
\centering
\includegraphics[width=1.0\linewidth]{images/img4}
\caption{}
\label{fig:img4}
\end{figure}


\begin{figure}[H]
\centering
\includegraphics[width=1.0\linewidth]{images/output}
\caption{Trajectory of the effector}
\label{fig:output}
\end{figure}


\endinput